%%%%%%%%%%%%%%%%%
% This is an sample CV template created using altacv.cls
% (v1.1.5, 1 December 2018) written by LianTze Lim (liantze@gmail.com). Now compiles with pdfLaTeX, XeLaTeX and LuaLaTeX.
%
%% It may be distributed and/or modified under the
%% conditions of the LaTeX Project Public License, either version 1.3
%% of this license or (at your option) any later version.
%% The latest version of this license is in
%%    http://www.latex-project.org/lppl.txt
%% and version 1.3 or later is part of all distributions of LaTeX
%% version 2003/12/01 or later.
%%%%%%%%%%%%%%%%


%% If you need to pass whatever options to xcolor
\PassOptionsToPackage{dvipsnames}{xcolor}

%% If you are using \orcid or academicons
%% icons, make sure you have the academicons
%% option here, and compile with XeLaTeX
%% or LuaLaTeX.
 %\documentclass[10pt,a4paper,academicons]{altacv}

%% Use the "normalphoto" option if you want a normal photo instead of cropped to a circle
% \documentclass[10pt,a4paper,normalphoto]{altacv}

\documentclass[10pt,a4paper,ragged2e]{altacv}
\usepackage[hidelinks]{hyperref}

%% AltaCV uses the fontawesome and academicon fonts
%% and packages.
%% See texdoc.net/pkg/fontawecome and http://texdoc.net/pkg/academicons for full list of symbols. You MUST compile with XeLaTeX or LuaLaTeX if you want to use academicons.

% Change the page layout if you need to
%\geometry{left=1cm,right=9cm,marginparwidth=6.8cm,marginparsep=1.2cm,top=1.25cm,bottom=1.25cm}
\geometry{left=1cm,right=9cm,marginparwidth=6.8cm,marginparsep=1.2cm,top=2cm,bottom=1.25cm}

% Change the font if you want to, depending on whether
% you're using pdflatex or xelatex/lualatex
\ifxetexorluatex
  % If using xelatex or lualatex:
  \setmainfont{Carlito}
\else
  % If using pdflatex:
  \usepackage[utf8]{inputenc}
  \usepackage[T1]{fontenc}
  \usepackage[default]{lato}
\fi

% Change the colours if you want to
\definecolor{Mulberry}{HTML}{72243D}
\definecolor{SlateGrey}{HTML}{2E2E2E}
\definecolor{LightGrey}{HTML}{666666}
\definecolor{prueba}{HTML}{E08B16}
\colorlet{heading}{Black}
\colorlet{separator}{Red}
\colorlet{accent}{prueba}
\colorlet{emphasis}{SlateGrey}
\colorlet{body}{LightGrey}

% Change the bullets for itemize and rating marker
% for \cvskill if you want to
\renewcommand{\itemmarker}{{\small\textbullet}}
\renewcommand{\ratingmarker}{\faCircle}

%% sample.bib contains your publications
\addbibresource{sample.bib}

\begin{document}
\name{Daniel Lopez Herranz}
\tagline{Pr\'acticas de operario}
%\tagline{Data scientist \& Software engineer}
%\tagline{Looking for a data scientist internship}
\tagline{Data Scientist \& Software Engineer}
%\photo{2.8cm}{Daniel(2)}
\personalinfo{%
  % Not all of these are required!
  % You can add your own with \printinfo{symbol}{detail}
  \email{\href{mailto:daniel.lopezh@student-cs.fr}{daniel.lopezh@student-cs.fr}}
  \phone{+33 769 44 71 16} \\
  \location{Paris, France}
  %% \location{Location, COUNTRY}
  \homepage{21 years-old (06/04/1998)} \\
  %% \twitter{@twitterhandle}
  %% \linkedin{linkedin.com/in/yourid}
  \linkedin{\href{www.linkedin.com/in/daniel-lopez-herranz-080b55172}{www.linkedin.com/in/daniel-lopez-herranz-080b55172}}
  %% \github{github.com/yourid}
  %% You MUST add the academicons option to \documentclass, then compile with LuaLaTeX or XeLaTeX, if you want to use \orcid or other academicons commands.
  %\orcid{orcid.org/0000-0000-0000-0000}
}

%% Make the header extend all the way to the right, if you want.
%\begin{fullwidth}
\makecvheader
%\end{fullwidth}

%\bigskip

%% Depending on your tastes, you may want to make fonts of itemize environments slightly smaller
% \AtBeginEnvironment{itemize}{\small}

%% Provide the file name containing the sidebar contents as an optional parameter to \cvsection.
%% You can always just use \marginpar{...} if you do
%% not need to align the top of the contents to any
%% \cvsection title in the "main" bar.

\cvsection[page1sidebar]{Experience}
\cvevent{Data scientist \& Software developer}{Paris Digital Lab}{2020}{Paris, France}
One year program for highly motivated students seeking to upgrade and improve their tech skills trough real proyects built for Companies.
\smallskip
\begin{itemize}
	\item \textbf{Flashbrand}: POC for a \textbf{video recognition software} capable of identifying emotions and giving feedback on that regard
\end{itemize}
\divider
\cvevent{\ Head of communication and logistics}{Beegames, CetraleSupelec club}{2018-2019}{Paris, France}
\begin{itemize}
	\item \textbf{Planning and organization} of a college sport event for students, pro players and companies
\end{itemize}
\divider
\cvevent{\ Member}{CS Design, CentraleSupelec club}{2018-2020}{Paris, France}
\begin{itemize}
	\item Graphic design: logos, posters and others using \textbf{Adobe software} for companies or other college clubs
\end{itemize}

%% \cvevent{Job Title 1}{Company 1}{Month 20XX -- Ongoing}{Location}
%% \begin{itemize}
%% \item Job description 1
%% \item Job description 2
%% \end{itemize}

%% \divider

%% \cvevent{Job Title 2}{Company 2}{Month 20XX -- Ongoing}{Location}
%% \begin{itemize}
%% \item Job description 1
%% \item Job description 2
%% \end{itemize}

\cvsection{Projects}

\cvevent{Emotion recognition software using facial expressions}{Flashbrand}{2020}{}
\begin{itemize}
	\item POC of an emotion recognition software using the algorithm developped by Fujitsu
\end{itemize}
\divider
\cvevent{\ Cybersecurity attack detection}{Cybersecurity @Centrale Supélec Rennes}{2018}{}
\begin{itemize}
	\item Part of an important cybersecurity project built in Java aiming at detecting possible attacks
\end{itemize}

\medskip

\cvsection{Hobbies \& Interests	}
%% \divider\smallskip
	\cvtag{\faHeartbeat} I enjoy playing basketball and doing sports in general \\
	%\cvtag{\faTerminal} Always happy to learn new technologies \\
	%\cvtag{\faSpinner} Always happy to learn new technologies \\
	\cvtag{\faChevronUp} Always happy to learn new technologies \\
	\cvtag{\faPlane} I love visiting new countries and discovering new cultures

%\cvsection{Hobbies \& Interests	}
%\begin{itemize}
	%\item \faHeartbeat $ $ Sport\\%: \medskip \quad Tous en général et le basket en particulier
%\divider
	%\item \faTerminal $ $ Computer science \\
%\divider
	%\item \faPlane $ $ Travelling \\
%\end{itemize}
%\divider
%\cvevent{\ Baccalauréat scientifique français et espagnol}{Lycée français Molière}{2016}{Saragosse, Espagne}

% Adapted from @Jake's answer from http://tex.stackexchange.com/a/82729/226
 %\wheelchart{outer radius}{inner radius}{
% comma-separated list of value/text width/color/detail}
 %\wheelchart{1.5cm}{0.5cm}{
   %6/8em/accent!30/{Sleep,\\beautiful sleep},
   %3/8em/accent!40/Hopeful novelist by night,
   %8/8em/accent!60/Daytime job,
   %2/10em/accent/Sports and relaxation,
   %5/6em/accent!20/Spending time with family
 %}

%%%%%%%%%%%%%%%%%%%%%%%%%%%%%%%%%%%%%%%%%%%%%%%%%%%%%%%%%%%%%%
%\clearpage
%\cvsection[page2sidebar]{Publications}

%\nocite{*}

%\printbibliography[heading=pubtype,title={\printinfo{\faBook}{Books}},type=book]

%\divider

%\printbibliography[heading=pubtype,title={\printinfo{\faFileTextO}{Journal Articles}},type=article]

%\divider

%\printbibliography[heading=pubtype,title={\printinfo{\faGroup}{Conference Proceedings}},type=inproceedings]

%% If the NEXT page doesn't start with a \cvsection but you'd
%% still like to add a sidebar, then use this command on THIS
%% page to add it. The optional argument lets you pull up the
%% sidebar a bit so that it looks aligned with the top of the
%% main column.
% \addnextpagesidebar[-1ex]{page3sidebar}


\end{document}
